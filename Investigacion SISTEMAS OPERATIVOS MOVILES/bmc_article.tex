%% BioMed_Central_Tex_Template_v1.06
%%                                      %
%  bmc_article.tex            ver: 1.06 %
%                                       %

%%IMPORTANT: do not delete the first line of this template
%%It must be present to enable the BMC Submission system to
%%recognise this template!!

%%%%%%%%%%%%%%%%%%%%%%%%%%%%%%%%%%%%%%%%%
%%                                     %%
%%  LaTeX template for BioMed Central  %%
%%     journal article submissions     %%
%%                                     %%
%%          <8 June 2012>              %%
%%                                     %%
%%                                     %%
%%%%%%%%%%%%%%%%%%%%%%%%%%%%%%%%%%%%%%%%%


%%%%%%%%%%%%%%%%%%%%%%%%%%%%%%%%%%%%%%%%%%%%%%%%%%%%%%%%%%%%%%%%%%%%%
%%                                                                 %%
%% For instructions on how to fill out this Tex template           %%
%% document please refer to Readme.html and the instructions for   %%
%% authors page on the biomed central website                      %%
%% http://www.biomedcentral.com/info/authors/                      %%
%%                                                                 %%
%% Please do not use \input{...} to include other tex files.       %%
%% Submit your LaTeX manuscript as one .tex document.              %%
%%                                                                 %%
%% All additional figures and files should be attached             %%
%% separately and not embedded in the \TeX\ document itself.       %%
%%                                                                 %%
%% BioMed Central currently use the MikTex distribution of         %%
%% TeX for Windows) of TeX and LaTeX.  This is available from      %%
%% http://www.miktex.org                                           %%
%%                                                                 %%
%%%%%%%%%%%%%%%%%%%%%%%%%%%%%%%%%%%%%%%%%%%%%%%%%%%%%%%%%%%%%%%%%%%%%

%%% additional documentclass options:
%  [doublespacing]
%  [linenumbers]   - put the line numbers on margins

%%% loading packages, author definitions

%\documentclass[twocolumn]{bmcart}% uncomment this for twocolumn layout and comment line below
\documentclass{bmcart}

%%% Load packages
%\usepackage{amsthm,amsmath}
%\RequirePackage{natbib}
%\RequirePackage[authoryear]{natbib}% uncomment this for author-year bibliography
%\RequirePackage{hyperref}
\usepackage[utf8]{inputenc} %unicode support
%\usepackage[applemac]{inputenc} %applemac support if unicode package fails
%\usepackage[latin1]{inputenc} %UNIX support if unicode package fails


%%%%%%%%%%%%%%%%%%%%%%%%%%%%%%%%%%%%%%%%%%%%%%%%%
%%                                             %%
%%  If you wish to display your graphics for   %%
%%  your own use using includegraphic or       %%
%%  includegraphics, then comment out the      %%
%%  following two lines of code.               %%
%%  NB: These line *must* be included when     %%
%%  submitting to BMC.                         %%
%%  All figure files must be submitted as      %%
%%  separate graphics through the BMC          %%
%%  submission process, not included in the    %%
%%  submitted article.                         %%
%%                                             %%
%%%%%%%%%%%%%%%%%%%%%%%%%%%%%%%%%%%%%%%%%%%%%%%%%


\def\includegraphic{}
\def\includegraphics{}



%%% Put your definitions there:
\startlocaldefs
\endlocaldefs


%%% Begin ...
\begin{document}

%%% Start of article front matter
\begin{frontmatter}

\begin{fmbox}
\dochead{Research}

%%%%%%%%%%%%%%%%%%%%%%%%%%%%%%%%%%%%%%%%%%%%%%
%%                                          %%
%% Enter the title of your article here     %%
%%                                          %%
%%%%%%%%%%%%%%%%%%%%%%%%%%%%%%%%%%%%%%%%%%%%%%

\title{Los sistema operativos para moviles.}

%%%%%%%%%%%%%%%%%%%%%%%%%%%%%%%%%%%%%%%%%%%%%%
%%                                          %%
%% Enter the authors here                   %%
%%                                          %%
%% Specify information, if available,       %%
%% in the form:                             %%
%%   <key>={<id1>,<id2>}                    %%
%%   <key>=                                 %%
%% Comment or delete the keys which are     %%
%% not used. Repeat \author command as much %%
%% as required.                             %%
%%                                          %%
%%%%%%%%%%%%%%%%%%%%%%%%%%%%%%%%%%%%%%%%%%%%%%



%%%%%%%%%%%%%%%%%%%%%%%%%%%%%%%%%%%%%%%%%%%%%%
%%                                          %%
%% Enter the authors' addresses here        %%
%%                                          %%
%% Repeat \address commands as much as      %%
%% required.                                %%
%%                                          %%
%%%%%%%%%%%%%%%%%%%%%%%%%%%%%%%%%%%%%%%%%%%%%%



%%%%%%%%%%%%%%%%%%%%%%%%%%%%%%%%%%%%%%%%%%%%%%
%%                                          %%
%% Enter short notes here                   %%
%%                                          %%
%% Short notes will be after addresses      %%
%% on first page.                           %%
%%                                          %%
%%%%%%%%%%%%%%%%%%%%%%%%%%%%%%%%%%%%%%%%%%%%%%

\begin{artnotes}
%\note{Sample of title note}     % note to the article
\note[id=n1]{Equal contributor} % note, connected to author
\end{artnotes}

\end{fmbox}% comment this for two column layout

%%%%%%%%%%%%%%%%%%%%%%%%%%%%%%%%%%%%%%%%%%%%%%
%%                                          %%
%% The Abstract begins here                 %%
%%                                          %%
%% Please refer to the Instructions for     %%
%% authors on http://www.biomedcentral.com  %%
%% and include the section headings         %%
%% accordingly for your article type.       %%
%%                                          %%
%%%%%%%%%%%%%%%%%%%%%%%%%%%%%%%%%%%%%%%%%%%%%%

\begin{abstractbox}

\begin{abstract} % abstract
	
	El objetivo del articulo es dar a conocer el sistema operativo que hoy conocemos como "Android", que es comun mente utilizado por la mayoria de las personas, asi como dar a conocer la influencia del sistema operativo Android en el mundo de los dispositivos móviles inteligentes que en la actualidad han evolucionando de manera considerable.

\end{abstract}

%%%%%%%%%%%%%%%%%%%%%%%%%%%%%%%%%%%%%%%%%%%%%%
%%                                          %%
%% The keywords begin here                  %%
%%                                          %%
%% Put each keyword in separate \kwd{}.     %%
%%                                          %%
%%%%%%%%%%%%%%%%%%%%%%%%%%%%%%%%%%%%%%%%%%%%%%



% MSC classifications codes, if any
%\begin{keyword}[class=AMS]
%\kwd[Primary ]{}
%\kwd{}
%\kwd[; secondary ]{}
%\end{keyword}

\end{abstractbox}
%
%\end{fmbox}% uncomment this for twcolumn layout

\end{frontmatter}

%%%%%%%%%%%%%%%%%%%%%%%%%%%%%%%%%%%%%%%%%%%%%%
%%                                          %%
%% The Main Body begins here                %%
%%                                          %%
%% Please refer to the instructions for     %%
%% authors on:                              %%
%% http://www.biomedcentral.com/info/authors%%
%% and include the section headings         %%
%% accordingly for your article type.       %%
%%                                          %%
%% See the Results and Discussion section   %%
%% for details on how to create sub-sections%%
%%                                          %%
%% use \cite{...} to cite references        %%
%%  \cite{koon} and                         %%
%%  \cite{oreg,khar,zvai,xjon,schn,pond}    %%
%%  \nocite{smith,marg,hunn,advi,koha,mouse}%%
%%                                          %%
%%%%%%%%%%%%%%%%%%%%%%%%%%%%%%%%%%%%%%%%%%%%%%

%%%%%%%%%%%%%%%%%%%%%%%%% start of article main body
% <put your article body there>

%%%%%%%%%%%%%%%%
%% Background %%
%%
\newpage
\section*{Introducción}

Los sistemas operativos son muy comúnmente consocios hoy en día, porque cualquier persona ha tenido contacto con ellos, desde una computadora personal hasta algún dispositivo móvil e inclusive en algunos coches muy modernos, que cuentas con esta herramienta fundamental en la computación en informática.
La información que estos manejan es muy rápido, que no tiene comparación con cualquier otra cosa, esto es gracias a la evolución del hardware, que ha tenido en las últimas décadas, enfocándose siempre a la superación de la velocidad de procesamiento, y en la sencilles de para poder utilizarlos. \\

Tan solo 10 años los dispositivos moviles han ganado mucho terreno gracias a que son sencillo y faciles de utilizar con una interface grafica amigable con el usuario , esto demuestra que este tipo de tecnología evoluciona de manera considerable, muy rápidamente y constantemente tenemos nuevos avances en este campo de la tecnologia.\\

Hoy en día hablamos de sistemas que están enfocados a darle una experiencia al usuario de su uso cotidiano, y dejar de lado los comandos puros de un sistema en esencia puramente la comunicación con los componentes físicos de la maquina han dejado de ser cosa que solo expertos conozcan y dar al público esta accesibilidad que hoy tenemos.\\

El sistema operativo de Google "Android", no es la excepción, ya que se considera como unos de los sistemas operativos actuales en el mercado con muy alta demanda, una utilidad increible y un alto nivel de desarrollo para el futuro proximo.
 %\cite{koon,oreg,khar,zvai,xjon,schn,pond,smith,marg,hunn,advi,koha,mouse}

\newpage

\section*{Definicion de un sistema operativo}

 El sistema operativo es un grupo de programas de proceso con las rutinas de
 control necesarias para mantener continuamente
 operativos dichos programas.
 
\subsection*{Objetivo primario de un sistema operativo:}

El objetivo principal de un sistema operativo es optimizar todos los recursos del sistema para soportar los requerimientos que se le pidan.


 
 
\section*{Sistemas Operativos para dispositivos móviles}
 
 Existen varios de ellos, si bien las más extendidas son Symbian, BlackBerry
 OS, Windows Mobile, y recientemente iPhone OS y el sistema móvil de Google,
 Android, además por supuesto de los dispositivos con sistema operativo Linux.

\subsection*{Symbian:}

Este es el sistema operativo para móviles más extendido entre “smartphones”, y por
tanto el que más aplicaciones para su sistema tiene desarrolladas. Actualmente
Symbian copa más del 65 del mercado de sistemas operativos.

\subsubsection*{Ventajas:}

Su principal virtud es la capacidad que tiene el sistema para adaptar e integrar todo
tipo de aplicaciones. Admite la integración de aplicaciones y, como sistema operativo,
ofrece las rutinas, los protocolos de comunicación, el control de archivos y los
servicios para el correcto funcionamiento de estas aplicaciones.

\subsection*{Windows Mobile:}

Windows Mobile es un sistema operativo escrito desde 0 y que hace uso de
algunas convenciones de la interfaz de usuario del Windows de siempre.

\subsubsection*{Ventajas:}

Una de las ventajas de Windows Mobile sobre sus competidores es que los
programadores pueden desarrollar aplicaciones para móviles utilizando los mismos
lenguajes y entornos que emplean con Windows para PC. En comparación, las
aplicaciones para Symbian necesitan más esfuerzo de desarrollo, aunque también
están optimizadas para cada modelo de teléfono.


\subsection*{Android:}

Android es un sistema operativo móvil basado en Linux y Java que ha sido
liberado bajo la licencia Apache versión 2.
El sistema busca, nuevamente, un modelo estandarizado de programación que
simplifique las labores de creación de aplicaciones móviles y normalice las
herramientas en el campo de la telefonía móvil. 

\subsubsection*{Ventajas:}

Su principal ventaja es que los programadores sólo tengan que desarrollar sus creaciones
una única vez y así ésta sea compatible con diferentes terminales gracias a la alta compatibilidad del leguage de programacion Java. 


\subsection*{ iPhone OS:}

Es una versión reducida de Mac OS X optimizada para los procesadores
ARM. Aunque oficialmente no se puede instalar ninguna aplicación que no esté
firmada por Apple ya existen formas de hacerlo, la vía oficial forma parte del iPhone
Developer Program (de pago) y hay que descargar el SKD que es gratuito.

\subsubsection*{Ventajas:}

iPhone dispone de un interfaz de usuario realmente interesante, lo unico que no satisface es la
cantidad de restricciones que tiene. para que iOS
triunfe mucho más es mejor liberar y dar libertad a su sistema.


\subsection*{ Blackberry OS:}

BlackBerry es un sistema operativo multitarea que está arrasando en la escena
empresarial, en especial por sus servicios para correo y teclado QWERTY.
Actualmente BlackBerry OS cuenta con un 11% del mercado.

\subsubsection*{Ventajas:}
 Este sistema operativo incorpora múltiples aplicaciones y programas
 que convierten a los dispositivos en completos organizadores de bolsillo con
 funciones de calendario, libreta de direcciones, bloc de notas, lista de tareas, entre
 otras.



\newpage  

\section*{“Smartphones” o teléfonos inteligentes}

	Un “smartphone” (teléfono inteligente en español) es un dispositivo electrónico que
	funciona como un teléfono móvil con características similares a las de una computadora
	personal. Es como un equivalente entre un teléfono móvil clásico y una PDA
	ya que permite hacer llamadas y enviar mensajes de texto como un móvil
	convencional pero además incluye características cercanas a las de una computadora. 
	Una característica importante de casi todos los teléfonos moviles es que
	permiten la instalación de programas (aplicaciones) para incrementar el procesamiento de datos y la
	conectividad. Estas aplicaciones pueden ser desarrolladas por el fabricante del
	dispositivo, por el operador o por un tercero.
	Los teléfonos inteligentes se distinguen por muchas características, entre las que
	destacan las pantallas táctiles, un sistema operativo así como la conectividad a
	Internet y el acceso al correo electrónico.

  
\subsection*{Sub-heading for section}
Text for this sub-heading \ldots
\subsubsection*{Sub-sub heading for section}
Text for this sub-sub-heading \ldots
\paragraph*{Sub-sub-sub heading for section}
Text for this sub-sub-sub-heading \ldots
In this section we examine the growth rate of the mean of $Z_0$, $Z_1$ and $Z_2$. In
addition, we examine a common modeling assumption and note the
importance of considering the tails of the extinction time $T_x$ in
studies of escape dynamics.
We will first consider the expected resistant population at $vT_x$ for
some $v>0$, (and temporarily assume $\alpha=0$)
%
\[
 E \bigl[Z_1(vT_x) \bigr]= E
\biggl[\mu T_x\int_0^{v\wedge
1}Z_0(uT_x)
\exp \bigl(\lambda_1T_x(v-u) \bigr)\,du \biggr].
\]
%
If we assume that sensitive cells follow a deterministic decay
$Z_0(t)=xe^{\lambda_0 t}$ and approximate their extinction time as
$T_x\approx-\frac{1}{\lambda_0}\log x$, then we can heuristically
estimate the expected value as
%
\begin{eqnarray}\label{eqexpmuts}
E\bigl[Z_1(vT_x)\bigr] &=& \frac{\mu}{r}\log x
\int_0^{v\wedge1}x^{1-u}x^{({\lambda_1}/{r})(v-u)}\,du
\nonumber\\
&=& \frac{\mu}{r}x^{1-{\lambda_1}/{\lambda_0}v}\log x\int_0^{v\wedge
1}x^{-u(1+{\lambda_1}/{r})}\,du
\nonumber\\
&=& \frac{\mu}{\lambda_1-\lambda_0}x^{1+{\lambda_1}/{r}v} \biggl(1-\exp \biggl[-(v\wedge1) \biggl(1+
\frac{\lambda_1}{r}\biggr)\log x \biggr] \biggr).
\end{eqnarray}
%
Thus we observe that this expected value is finite for all $v>0$ (also see \cite{koon,khar,zvai,xjon,marg}).
%\nocite{oreg,schn,pond,smith,marg,hunn,advi,koha,mouse}

%%%%%%%%%%%%%%%%%%%%%%%%%%%%%%%%%%%%%%%%%%%%%%
%%                                          %%
%% Backmatter begins here                   %%
%%                                          %%
%%%%%%%%%%%%%%%%%%%%%%%%%%%%%%%%%%%%%%%%%%%%%%

\begin{backmatter}

\section*{Competing interests}
  The authors declare that they have no competing interests.

\section*{Author's contributions}
    Text for this section \ldots

\section*{Acknowledgements}
  Text for this section \ldots
%%%%%%%%%%%%%%%%%%%%%%%%%%%%%%%%%%%%%%%%%%%%%%%%%%%%%%%%%%%%%
%%                  The Bibliography                       %%
%%                                                         %%
%%  Bmc_mathpys.bst  will be used to                       %%
%%  create a .BBL file for submission.                     %%
%%  After submission of the .TEX file,                     %%
%%  you will be prompted to submit your .BBL file.         %%
%%                                                         %%
%%                                                         %%
%%  Note that the displayed Bibliography will not          %%
%%  necessarily be rendered by Latex exactly as specified  %%
%%  in the online Instructions for Authors.                %%
%%                                                         %%
%%%%%%%%%%%%%%%%%%%%%%%%%%%%%%%%%%%%%%%%%%%%%%%%%%%%%%%%%%%%%

% if your bibliography is in bibtex format, use those commands:
\bibliographystyle{bmc-mathphys} % Style BST file (bmc-mathphys, vancouver, spbasic).
\bibliography{bmc_article}      % Bibliography file (usually '*.bib' )
% for author-year bibliography (bmc-mathphys or spbasic)
% a) write to bib file (bmc-mathphys only)
% @settings{label, options="nameyear"}
% b) uncomment next line
%\nocite{label}

% or include bibliography directly:
% \begin{thebibliography}
% \bibitem{b1}
% \end{thebibliography}

%%%%%%%%%%%%%%%%%%%%%%%%%%%%%%%%%%%
%%                               %%
%% Figures                       %%
%%                               %%
%% NB: this is for captions and  %%
%% Titles. All graphics must be  %%
%% submitted separately and NOT  %%
%% included in the Tex document  %%
%%                               %%
%%%%%%%%%%%%%%%%%%%%%%%%%%%%%%%%%%%

%%
%% Do not use \listoffigures as most will included as separate files

\section*{Figures}
  \begin{figure}[h!]
  \caption{\csentence{Sample figure title.}
      A short description of the figure content
      should go here.}
      \end{figure}

\begin{figure}[h!]
  \caption{\csentence{Sample figure title.}
      Figure legend text.}
      \end{figure}

%%%%%%%%%%%%%%%%%%%%%%%%%%%%%%%%%%%
%%                               %%
%% Tables                        %%
%%                               %%
%%%%%%%%%%%%%%%%%%%%%%%%%%%%%%%%%%%

%% Use of \listoftables is discouraged.
%%
\section*{Tables}
\begin{table}[h!]
\caption{Sample table title. This is where the description of the table should go.}
      \begin{tabular}{cccc}
        \hline
           & B1  &B2   & B3\\ \hline
        A1 & 0.1 & 0.2 & 0.3\\
        A2 & ... & ..  & .\\
        A3 & ..  & .   & .\\ \hline
      \end{tabular}
\end{table}

%%%%%%%%%%%%%%%%%%%%%%%%%%%%%%%%%%%
%%                               %%
%% Additional Files              %%
%%                               %%
%%%%%%%%%%%%%%%%%%%%%%%%%%%%%%%%%%%

\section*{Additional Files}
  \subsection*{Additional file 1 --- Sample additional file title}
    Additional file descriptions text (including details of how to
    view the file, if it is in a non-standard format or the file extension).  This might
    refer to a multi-page table or a figure.

  \subsection*{Additional file 2 --- Sample additional file title}
    Additional file descriptions text.


\end{backmatter}
\end{document}
