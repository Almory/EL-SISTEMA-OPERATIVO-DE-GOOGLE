%% BioMed_Central_Tex_Template_v1.06
%%                                      %
%  bmc_article.tex            ver: 1.06 %
%                                       %

%%IMPORTANT: do not delete the first line of this template
%%It must be present to enable the BMC Submission system to
%%recognise this template!!

%%%%%%%%%%%%%%%%%%%%%%%%%%%%%%%%%%%%%%%%%
%%                                     %%
%%  LaTeX template for BioMed Central  %%
%%     journal article submissions     %%
%%                                     %%
%%          <8 June 2012>              %%
%%                                     %%
%%                                     %%
%%%%%%%%%%%%%%%%%%%%%%%%%%%%%%%%%%%%%%%%%


%%%%%%%%%%%%%%%%%%%%%%%%%%%%%%%%%%%%%%%%%%%%%%%%%%%%%%%%%%%%%%%%%%%%%
%%                                                                 %%
%% For instructions on how to fill out this Tex template           %%
%% document please refer to Readme.html and the instructions for   %%
%% authors page on the biomed central website                      %%
%% http://www.biomedcentral.com/info/authors/                      %%
%%                                                                 %%
%% Please do not use \input{...} to include other tex files.       %%
%% Submit your LaTeX manuscript as one .tex document.              %%
%%                                                                 %%
%% All additional figures and files should be attached             %%
%% separately and not embedded in the \TeX\ document itself.       %%
%%                                                                 %%
%% BioMed Central currently use the MikTex distribution of         %%
%% TeX for Windows) of TeX and LaTeX.  This is available from      %%
%% http://www.miktex.org                                           %%
%%                                                                 %%
%%%%%%%%%%%%%%%%%%%%%%%%%%%%%%%%%%%%%%%%%%%%%%%%%%%%%%%%%%%%%%%%%%%%%

%%% additional documentclass options:
%  [doublespacing]
%  [linenumbers]   - put the line numbers on margins

%%% loading packages, author definitions

%\documentclass[twocolumn]{bmcart}% uncomment this for twocolumn layout and comment line below
\documentclass{bmcart}

%%% Load packages
%\usepackage{amsthm,amsmath}
%\RequirePackage{natbib}
%\RequirePackage[authoryear]{natbib}% uncomment this for author-year bibliography
%\RequirePackage{hyperref}
\usepackage[utf8]{inputenc} %unicode support
%\usepackage[applemac]{inputenc} %applemac support if unicode package fails
%\usepackage[latin1]{inputenc} %UNIX support if unicode package fails


%%%%%%%%%%%%%%%%%%%%%%%%%%%%%%%%%%%%%%%%%%%%%%%%%
%%                                             %%
%%  If you wish to display your graphics for   %%
%%  your own use using includegraphic or       %%
%%  includegraphics, then comment out the      %%
%%  following two lines of code.               %%
%%  NB: These line *must* be included when     %%
%%  submitting to BMC.                         %%
%%  All figure files must be submitted as      %%
%%  separate graphics through the BMC          %%
%%  submission process, not included in the    %%
%%  submitted article.                         %%
%%                                             %%
%%%%%%%%%%%%%%%%%%%%%%%%%%%%%%%%%%%%%%%%%%%%%%%%%


\def\includegraphic{}
\def\includegraphics{}



%%% Put your definitions there:
\startlocaldefs
\endlocaldefs


%%% Begin ...
\begin{document}

%%% Start of article front matter
\begin{frontmatter}

\begin{fmbox}
\dochead{Investigación}

%%%%%%%%%%%%%%%%%%%%%%%%%%%%%%%%%%%%%%%%%%%%%%
%%                                          %%
%% Enter the title of your article here     %%
%%                                          %%
%%%%%%%%%%%%%%%%%%%%%%%%%%%%%%%%%%%%%%%%%%%%%%

\title{Los sistema operativos para moviles (Android).}

%%%%%%%%%%%%%%%%%%%%%%%%%%%%%%%%%%%%%%%%%%%%%%
%%                                          %%
%% Enter the authors here                   %%
%%                                          %%
%% Specify information, if available,       %%
%% in the form:                             %%
%%   <key>={<id1>,<id2>}                    %%
%%   <key>=                                 %%
%% Comment or delete the keys which are     %%
%% not used. Repeat \author command as much %%
%% as required.                             %%
%%                                          %%
%%%%%%%%%%%%%%%%%%%%%%%%%%%%%%%%%%%%%%%%%%%%%%



%%%%%%%%%%%%%%%%%%%%%%%%%%%%%%%%%%%%%%%%%%%%%%
%%                                          %%
%% Enter the authors' addresses here        %%
%%                                          %%
%% Repeat \address commands as much as      %%
%% required.                                %%
%%                                          %%
%%%%%%%%%%%%%%%%%%%%%%%%%%%%%%%%%%%%%%%%%%%%%%



%%%%%%%%%%%%%%%%%%%%%%%%%%%%%%%%%%%%%%%%%%%%%%
%%                                          %%
%% Enter short notes here                   %%
%%                                          %%
%% Short notes will be after addresses      %%
%% on first page.                           %%
%%                                          %%
%%%%%%%%%%%%%%%%%%%%%%%%%%%%%%%%%%%%%%%%%%%%%%

\begin{artnotes}
%\note{Sample of title note}     % note to the article
\note[id=n1]{Equal contributor} % note, connected to author
\end{artnotes}

\end{fmbox}% comment this for two column layout

%%%%%%%%%%%%%%%%%%%%%%%%%%%%%%%%%%%%%%%%%%%%%%
%%                                          %%
%% The Abstract begins here                 %%
%%                                          %%
%% Please refer to the Instructions for     %%
%% authors on http://www.biomedcentral.com  %%
%% and include the section headings         %%
%% accordingly for your article type.       %%
%%                                          %%
%%%%%%%%%%%%%%%%%%%%%%%%%%%%%%%%%%%%%%%%%%%%%%

\begin{abstractbox}

\begin{abstract} % abstract

	El objetivo del articulo es dar a conocer el sistema operativo que hoy conocemos como "Android", que es comun mente utilizado por la mayoria de las personas, asi como dar a conocer la influencia del sistema operativo Android en el mundo de los dispositivos móviles inteligentes que en la actualidad han evolucionando de manera considerable.


\end{abstract}

%%%%%%%%%%%%%%%%%%%%%%%%%%%%%%%%%%%%%%%%%%%%%%
%%                                          %%
%% The keywords begin here                  %%
%%                                          %%
%% Put each keyword in separate \kwd{}.     %%
%%                                          %%
%%%%%%%%%%%%%%%%%%%%%%%%%%%%%%%%%%%%%%%%%%%%%%



% MSC classifications codes, if any
%\begin{keyword}[class=AMS]
%\kwd[Primary ]{}
%\kwd{}
%\kwd[; secondary ]{}
%\end{keyword}

\end{abstractbox}
%
%\end{fmbox}% uncomment this for twcolumn layout

\end{frontmatter}

%%%%%%%%%%%%%%%%%%%%%%%%%%%%%%%%%%%%%%%%%%%%%%
%%                                          %%
%% The Main Body begins here                %%
%%                                          %%
%% Please refer to the instructions for     %%
%% authors on:                              %%
%% http://www.biomedcentral.com/info/authors%%
%% and include the section headings         %%
%% accordingly for your article type.       %%
%%                                          %%
%% See the Results and Discussion section   %%
%% for details on how to create sub-sections%%
%%                                          %%
%% use \cite{...} to cite references        %%
%%  \cite{koon} and                         %%
%%  \cite{oreg,khar,zvai,xjon,schn,pond}    %%
%%  \nocite{smith,marg,hunn,advi,koha,mouse}%%
%%                                          %%
%%%%%%%%%%%%%%%%%%%%%%%%%%%%%%%%%%%%%%%%%%%%%%

%%%%%%%%%%%%%%%%%%%%%%%%% start of article main body
% <put your article body there>

%%%%%%%%%%%%%%%%
%% Background %%
%%

\section*{Introducción}

Los sistemas operativos son muy comúnmente consocios hoy en día, porque cualquier persona ha tenido contacto con ellos, desde una computadora personal hasta algún dispositivo móvil e inclusive en algunos coches muy modernos, que cuentas con esta herramienta fundamental en la computación en informática.
La información que estos manejan es muy rápido, que no tiene comparación con cualquier otra cosa, esto es gracias a la evolución del hardware, que ha tenido en las últimas décadas, enfocándose siempre a la superación de la velocidad de procesamiento, y en la sencilles de para poder utilizarlos. \\

Tan solo 10 años los dispositivos moviles han ganado mucho terreno gracias a que son sencillo y faciles de utilizar con una interface grafica amigable con el usuario , esto demuestra que este tipo de tecnología evoluciona de manera considerable, muy rápidamente y constantemente tenemos nuevos avances en este campo de la tecnologia.\\

Hoy en día hablamos de sistemas que están enfocados a darle una experiencia al usuario de su uso cotidiano, y dejar de lado los comandos puros de un sistema en esencia puramente la comunicación con los componentes físicos de la maquina han dejado de ser cosa que solo expertos conozcan y dar al público esta accesibilidad que hoy tenemos.\\

El sistema operativo de Google "Android", no es la excepción, ya que se considera como unos de los sistemas operativos actuales en el mercado con muy alta demanda, una utilidad increible y un alto nivel de desarrollo para el futuro proximo.
%\cite{koon,oreg,khar,zvai,xjon,schn,pond,smith,marg,hunn,advi,koha,mouse}

\newpage


\section*{Objetivos}

Conocer los distintos sistemas operativos para moviles que existen en la actualidad, y profundizar en la evolucion en especifico del sistema operativo de google Android, para conocer su caracteristicas.\newline

Mostrar los sistemas operativos, con ello se podrá comprender de una manera más amplia y fácil los sistemas para moviles actuales, ya que cada vez son más fáciles de utilizar para el usuario más que nada.Aprender sobre sus debilidades, dando a conocer la manera en la que este sistema maneja la información, ya que la manera puede ser vulnerable o por lo contrario muy bien protegida.

\section*{Justificación}

Analizando los distintos sistemas operativos, se pretende hacer una agrupación de sus diferencias, logrando así, colocarlos de una manera práctica en cuanto a su uso y utilidad, con esto se pretende distinguir sus diferencias.\newline
Hoy en día se sabe que los sistemas operativos con los que se cuenta no son iguales que los de hace 20 años, pero sí se consideran la base de su evolución, y cómo fueron surgiendo de ellos, con esto se logra una explicación aún más detallada y con referencia a sus funcionalidades y el estado actual de ellos.\newline
El tema está relacionado con la vida diaria en la cual forma parte importante y vital de cualquier individuo común relacionado con el área, como los futuros desarrolladores de software. Se eligió el tema de sistemas operativos, ya que actualmente  la mayoría de las personas tienen contacto con el mismo, usando sus dispositivos móviles y/o computadoras, para lo cual se  hace de suma importancia el que tengan el conocimiento,  debido a que  les permite estar más orientados y enterados de los factores que involucran el funcionamiento de la tecnología y todo lo que les conlleva.

\newpage

\section*{Definición de un sistema operativo}

El sistema operativo es un grupo de programas de proceso con las rutinas de
control necesarias para mantener continuamente
operativos dichos programas.

\subsection*{Objetivo primario de un sistema operativo:}

El objetivo principal de un sistema operativo es optimizar todos los recursos del sistema para soportar los requerimientos que se le pidan.\cite{tanenbaum1998sistemas}\cite{luis2001sistemas}


\section*{“Smartphones” o teléfonos inteligentes}

Un “smartphone” (teléfono inteligente en español) es un dispositivo electrónico que
funciona como un teléfono móvil con características similares a las de una computadora
personal. Es como un equivalente entre un teléfono móvil clásico y una PDA
ya que permite hacer llamadas y enviar mensajes de texto como un móvil
convencional pero además incluye características cercanas a las de una computadora. 
Una característica importante de casi todos los teléfonos moviles es que
permiten la instalación de programas (aplicaciones) para incrementar el procesamiento de datos y la
conectividad. Estas aplicaciones pueden ser desarrolladas por el fabricante del
dispositivo, por el operador o por un tercero.
Los teléfonos inteligentes se distinguen por muchas características, entre las que
destacan las pantallas táctiles, un sistema operativo así como la conectividad a
Internet y el acceso al correo electrónico.\cite{del2009sistemas}


\section*{Localización en un sistema de computo}

El sistema operativo se encuentra por encima del lenguage maquina, y este es el principal traductor y conversor para el mismo.
Inferior a este se encuentra el hardware, y en el nivel mas bajo se encuentran los dispositivos fisicos hechos con circuitos integrados, cables, fuentes de potencia, etc.\cite{tanenbaum2003sistemas}

\newpage
\section*{Principales Sistemas Operativos para dispositivos móviles}

Existen varios de ellos, si bien las más extendidas son Symbian, BlackBerry
OS, Windows Mobile, y recientemente iPhone OS y el sistema móvil de Google,
Android, además por supuesto de los dispositivos con sistema operativo Linux.\cite{alonsodispositivos}\cite{del2009sistemas}

\subsection*{Symbian:}

Este es el sistema operativo para móviles más extendido entre “smartphones”, y por
tanto el que más aplicaciones para su sistema tiene desarrolladas. Actualmente
Symbian copa más del 65 del mercado de sistemas operativos.\cite{del2009sistemas}

\subsubsection*{Ventajas:}

Su principal virtud es la capacidad que tiene el sistema para adaptar e integrar todo
tipo de aplicaciones. Admite la integración de aplicaciones y, como sistema operativo,
ofrece las rutinas, los protocolos de comunicación, el control de archivos y los
servicios para el correcto funcionamiento de estas aplicaciones.

\subsection*{Windows Mobile:}

Windows Mobile es un sistema operativo escrito desde 0 y que hace uso de
algunas convenciones de la interfaz de usuario del Windows de siempre.\cite{del2009sistemas}

\subsubsection*{Ventajas:}

Una de las ventajas de Windows Mobile sobre sus competidores es que los
programadores pueden desarrollar aplicaciones para móviles utilizando los mismos
lenguajes y entornos que emplean con Windows para PC. En comparación, las
aplicaciones para Symbian necesitan más esfuerzo de desarrollo, aunque también
están optimizadas para cada modelo de teléfono.


\subsection*{Android:}

Android es un sistema operativo móvil basado en Linux y Java que ha sido
liberado bajo la licencia Apache versión 2.
El sistema busca, nuevamente, un modelo estandarizado de programación que
simplifique las labores de creación de aplicaciones móviles y normalice las
herramientas en el campo de la telefonía móvil.\cite{del2009sistemas}\cite{polanco2011android}

\subsubsection*{Ventajas:}

Su principal ventaja es que los programadores sólo tengan que desarrollar sus creaciones
una única vez y así ésta sea compatible con diferentes terminales gracias a la alta compatibilidad del leguage de programacion Java. 


\subsection*{ iPhone OS:}

Es una versión reducida de Mac OS X optimizada para los procesadores
ARM. Aunque oficialmente no se puede instalar ninguna aplicación que no esté
firmada por Apple ya existen formas de hacerlo, la vía oficial forma parte del iPhone
Developer Program (de pago) y hay que descargar el SKD que es gratuito.\cite{del2009sistemas}

\subsubsection*{Ventajas:}

iPhone dispone de un interfaz de usuario realmente interesante, lo unico que no satisface es la
cantidad de restricciones que tiene. para que iOS
triunfe mucho más es mejor liberar y dar libertad a su sistema.


\subsection*{ Blackberry OS:}

BlackBerry es un sistema operativo multitarea que está arrasando en la escena
empresarial, en especial por sus servicios para correo y teclado QWERTY.
Actualmente BlackBerry OS cuenta con un 11\% del mercado.\cite{del2009sistemas}

\subsubsection*{Ventajas:}
Este sistema operativo incorpora múltiples aplicaciones y programas
que convierten a los dispositivos en completos organizadores de bolsillo con
funciones de calendario, libreta de direcciones, bloc de notas, lista de tareas, entre
otras.



\newpage  

\section*{Android}

\subsection*{Historia}
En sus inicios, únicamente trascendió que la actividad de la empresa se
centraba en “el desarrollo de software para teléfonos móviles“.\newline
Android Inc. pasó casi dos años trabajando “en la sombra”, hasta que Google comenzó
a “reclutar” a fuerza de talonario a algunas “startup” (término que se refiere a nuevas
compañías con un futuro prometedor) del sector móvil, con la clara intención de replicar su
éxito de la Web en el futuro de las telecomunicaciones inalámbricas.

En Mayo del mismo año Google se hacía con Dodgeball, la empresa que desarrolló un
sistema de red social y posicionamiento móvil que, una vez integrada en la estructura
empresarial de los chicos de Mountain View, cesó su actividad en 2009 para dar paso a Google
Latitude. Llegó el mes de agosto y le tocó el turno a Android Inc., la fecha clave para llegar a
entender mejor el éxito de Android es el 5 de noviembre de 2007. Ese día se fundaba la OHA
(Open Handset Alliance), una alianza comercial de 35 componentes iniciales liderada por
Google, que contaba con fabricantes de terminales móviles, operadores de
telecomunicaciones, fabricantes de chips y desarrolladores de software. El mismo día se dio a
conocer por vez primera lo que hoy conocemos como Android, una plataforma de código
abierto para móviles que se presentaba con la garantía de estar basada en el sistema operativo
Linux. 

En Mayo del mismo año Google se hacía con Dodgeball, la empresa que desarrolló un
sistema de red social y posicionamiento móvil que, una vez integrada en la estructura
empresarial de los chicos de Mountain View, cesó su actividad en 2009 para dar paso a Google
Latitude. Llegó el mes de agosto y le tocó el turno a Android Inc., la fecha clave para llegar a
entender mejor el éxito de Android es el 5 de noviembre de 2007. Ese día se fundaba la OHA
(Open Handset Alliance), una alianza comercial de 35 componentes iniciales liderada por
Google, que contaba con fabricantes de terminales móviles, operadores de
telecomunicaciones, fabricantes de chips y desarrolladores de software. El mismo día se dio a
conocer por vez primera lo que hoy conocemos como Android, una plataforma de código
abierto para móviles que se presentaba con la garantía de estar basada en el sistema operativo
Linux. \cite{herraiz2012android}\cite{baez1997introduccion}\cite{benbourahala2013android}\newline


\subsection*{¿Que es Android?}

Android es un sistema operativo basado en GNU/Linux de código abierto bajo licencia Apache, el cual permite la creación principalmente de aplicaciones para dispositivos móviles teléfonos inteligentes, tablets, reproductores MP3, notebook, y otros desarrollado por Google  y actualmente liderado por el grupo Open Handset Alliance, en el cual se agrupan varias compañías del sector, entre las cuales se encuentran: Google, Samsung, HTC, Dell, Intel, Qualcomm, Motorola, LG, Telefónica, T-Mobile, Nvidia.\cite{vanegas2014android}\cite{polanco2011android}\cite{baez1997introduccion}\cite{benbourahala2013android}


\subsection*{Arquitectura:}

La arquitectura de Android, se conforma por cuatro capas o niveles que le permiten al programador la creación de aplicaciones. Su distribución ayuda a acceder a las diferentes capas por medio de librerías y cada capa utiliza los elementos de la capa inferior para realizar sus funciones, por eso, su arquitectura es tipo pila. La arquitectura del sistema operativo Android se puede aprecia en la siguiente gráfica:\cite{vanegas2014android}\newline

Kernel de Linux.
El núcleo actúa como una capa de abstracción entre el hardware y el resto de las capas de la arquitectura. 
\newline


Librerías.
La siguiente capa que se sitúa justo sobre el kernel la componen las bibliotecas nativas de Android, también llamadas librerías. Están escritas en C o C++ y compiladas para la arquitectura hardware específica del teléfono.
\newline

Entorno de ejecución.
Aquí encontramos las librerías con la funcionalidades habituales de Java así como otras específicas de Android.
\newline

Framework de aplicaciones.
La siguiente capa está formada por todas las clases y servicios que utilizan directamente las aplicaciones para realizar sus funciones. 
\newline

\subsection*{Versiones:}
Cupcake: Android Version 1.5\newline
Donut: Android Version 1.6 \newline
Eclair: Android Version 2.0/2.1 \newline
Froyo: Android Version 2.2\newline
Ginger Bread: Android Version 2.3 \newline
Honey Comb: Android Version 3.0/3.4\newline
Ice Cream Sandwich: Android Version 4.0 \newline
Jelly Bean: Android Version 4.1 \newline
Jelly Bean (Gummy Bear): Android Version 4.2 \newline
Jelly Bean: Android Version 4.3 \newline
KitKat (Dugger): Android Version 4.4 \newline
Lollipop: Android Version 5.0\newline
Marshmallow: Android Version 6.0

\newpage
\subsection*{Características}

\subsubsection*{Diseño de dispositivo}
La plataforma es adaptable a pantallas de mayor resolución, VGA, biblioteca de gráficos 2D, biblioteca de gráficos 3D basada en las especificaciones de la OpenGL ES 2.0 y diseño de teléfonos tradicionales.

\subsubsection*{Conectividad}
Android soporta las siguientes tecnologías de conectividad: GSM/EDGE, IDEN, CDMA, EV-DO, UMTS, Bluetooth, Wi-Fi, LTE, HSDPA, HSPA+, NFC y WiMAX.GPRS, UMTS y HSDPA+.

\subsubsection*{Entorno de desarrollo}

Incluye un emulador de dispositivos, herramientas para depuración de memoria y análisis del rendimiento del software. Inicialmente el entorno de desarrollo integrado (IDE) utilizado era Eclipse con el plugin de Herramientas de Desarrollo de Android (ADT). Ahora se considera como entorno oficial Android Studio, descargable desde la página oficial de desarrolladores de Android.

\subsubsection*{Multi-táctil}

Android tiene soporte nativo para pantallas capacitivas con soporte multi-táctil que inicialmente hicieron su aparición en dispositivos como el HTC Hero. La funcionalidad fue originalmente desactivada a nivel de kernel (posiblemente para evitar infringir patentes de otras compañías). Más tarde, Google publicó una actualización para el Nexus One y el Motorola Droid que activa el soporte multi-táctil de forma nativa.

\subsubsection*{Multitarea}
Multitarea real de aplicaciones está disponible, es decir, las aplicaciones que no estén ejecutándose en primer plano reciben ciclos de reloj.

\subsubsection*{Tethering}

Android soporta tethering, que permite al teléfono ser usado como un punto de acceso alámbrico o inalámbrico (todos los teléfonos desde la versión 2.2, no oficial en teléfonos con versión 1.6 o inferiores mediante aplicaciones disponibles en Google Play (por ejemplo PdaNet). Para permitir a un PC usar la conexión de datos del móvil android se podría requerir la instalación de software adicional.

\subsubsection*{Google Play}

Google Play es un catálogo de aplicaciones gratuitas o de pago en el que pueden ser descargadas e instaladas en dispositivos Android sin la necesidad de un PC.

\subsubsection*{Soporte para hardware adicional}

Android soporta cámaras de fotos, de vídeo, pantallas táctiles, GPS, acelerómetros, giroscopios, magnetómetros, sensores de proximidad y de presión, sensores de luz, gamepad, termómetro, aceleración por GPU 2D y 3D.

\newpage
\subsection*{Seguridad, privacidad y vigilancia}

Según un estudio de Symantec de 2013, demuestra que en comparación con iOS, Android es un sistema explícitamente menos vulnerable. El estudio en cuestión habla de 13 vulnerabilidades graves para Android y 387 vulnerabilidades graves para iOS. El estudio también habla de los ataques en ambas plataformas, en este caso Android se queda con 113 ataques nuevos en 2012 a diferencia de iOS que se queda en 1 solo ataque. Incluso así Google y Apple se empeñan cada vez más en hacer sus sistemas operativos más seguros incorporando más seguridad tanto en sus sistemas operativos como en sus mercados oficiales.

\subsection*{Historial de actualizaciones}

Android ha visto numerosas actualizaciones desde su liberación inicial. Estas actualizaciones al sistema operativo base típicamente arreglan bugs y agregan nuevas funciones. Generalmente cada actualización del sistema operativo Android es desarrollada bajo un nombre en código de un elemento relacionado con dulces en orden alfabético.\newline

La reiterada aparición de nuevas versiones que, en muchos casos, no llegan a funcionar correctamente en el hardware diseñado para versiones previas, hacen que Android sea considerado uno de los elementos promotores de la obsolescencia programada.\newline

Android ha sido criticado muchas veces por la fragmentación que sufren sus terminales al no ser soportado con actualizaciones constantes por los distintos fabricantes. Se creyó que esta situación cambiaría tras un anuncio de Google en el que comunicó que los fabricantes se comprometerán a aplicar actualizaciones al menos 18 meses desde su salida al mercado, pero esto al final nunca se concretó y el proyecto se canceló. Google actualmente intenta enmendar el problema con su plataforma actualizable Servicios de Google Play (que funciona en Android 2.2 y posteriores), separando todas las aplicaciones posibles del sistema (como Maps, el teclado, Youtube, Drive, e incluso la propia Play Store) para poder actualizarlas de manera independiente, e incluyendo la menor cantidad posible de novedades en las nuevas versiones de Android.


\newpage
\section*{Conlcusion}

Para concluir, Android es un sistema que ha surgido recientemente pero que ha tenido un largo trayecto de varios años.\newline
Es un sistema muy sólido que tiene mucho que dar al mundo y que es el futuro de la tecnología que conocemos.\newline
Y que a comparación de su competencia tiene mayor aceptabilidad por la sociedad, eso hace que siga creciendo como lo hace ahora y que el servicio que ofrece cada vez sea mejor, claro que en su momento ha fallado pero no es sino para mejorar su función.\newline
Sin duda un sistema que muy pocos pueden superar debido a su fácil accesibilidad y funcionamiento, que lo hace fácil de entender.
Esperaremos aún mejores cosas de el en el futuro.\newline
Si bien los distintos tipos de sistemas con los que se cuenta hoy en día son muy diversos, el principal objetivo de todos es poder lograr dar una experiencia al usuario en cuanto a su uso y manejo, y que toda persona pueda y sepa utilizarlos de una forma más adecuada.


\newpage

%%%%%%%%%%%%%%%%%%%%%%%%%%%%%%%%%%%%%%%%%%%%%%
%%                                          %%
%% Backmatter begins here                   %%
%%                                          %%
%%%%%%%%%%%%%%%%%%%%%%%%%%%%%%%%%%%%%%%%%%%%%%

\begin{backmatter}


%%%%%%%%%%%%%%%%%%%%%%%%%%%%%%%%%%%%%%%%%%%%%%%%%%%%%%%%%%%%%
%%                  The Bibliography                       %%
%%                                                         %%
%%  Bmc_mathpys.bst  will be used to                       %%
%%  create a .BBL file for submission.                     %%
%%  After submission of the .TEX file,                     %%
%%  you will be prompted to submit your .BBL file.         %%
%%                                                         %%
%%                                                         %%
%%  Note that the displayed Bibliography will not          %%
%%  necessarily be rendered by Latex exactly as specified  %%
%%  in the online Instructions for Authors.                %%
%%                                                         %%
%%%%%%%%%%%%%%%%%%%%%%%%%%%%%%%%%%%%%%%%%%%%%%%%%%%%%%%%%%%%%

% if your bibliography is in bibtex format, use those commands:
\bibliographystyle{bmc-mathphys} % Style BST file (bmc-mathphys, vancouver, spbasic).
\bibliography{bmc_article}      % Bibliography file (usually '*.bib' )
% for author-year bibliography (bmc-mathphys or spbasic)
% a) write to bib file (bmc-mathphys only)
% @settings{label, options="nameyear"}
% b) uncomment next line
%\nocite{label}

% or include bibliography directly:
% \begin{thebibliography}
% \bibitem{b1}
% \end{thebibliography}

%%%%%%%%%%%%%%%%%%%%%%%%%%%%%%%%%%%
%%                               %%
%% Figures                       %%
%%                               %%
%% NB: this is for captions and  %%
%% Titles. All graphics must be  %%
%% submitted separately and NOT  %%
%% included in the Tex document  %%
%%                               %%
%%%%%%%%%%%%%%%%%%%%%%%%%%%%%%%%%%%

%%
%% Do not use \listoffigures as most will included as separate files


%%%%%%%%%%%%%%%%%%%%%%%%%%%%%%%%%%%
%%                               %%
%% Tables                        %%
%%                               %%
%%%%%%%%%%%%%%%%%%%%%%%%%%%%%%%%%%%

%% Use of \listoftables is discouraged.
%%


%%%%%%%%%%%%%%%%%%%%%%%%%%%%%%%%%%%
%%                               %%
%% Additional Files              %%
%%                               %%
%%%%%%%%%%%%%%%%%%%%%%%%%%%%%%%%%%%


\end{backmatter}
\end{document}
