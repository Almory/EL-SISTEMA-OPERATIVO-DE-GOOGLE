%% BioMed_Central_Tex_Template_v1.06
%%                                      %
%  bmc_article.tex            ver: 1.06 %
%                                       %

%%IMPORTANT: do not delete the first line of this template
%%It must be present to enable the BMC Submission system to
%%recognise this template!!

%%%%%%%%%%%%%%%%%%%%%%%%%%%%%%%%%%%%%%%%%
%%                                     %%
%%  LaTeX template for BioMed Central  %%
%%     journal article submissions     %%
%%                                     %%
%%          <8 June 2012>              %%
%%                                     %%
%%                                     %%
%%%%%%%%%%%%%%%%%%%%%%%%%%%%%%%%%%%%%%%%%


%%%%%%%%%%%%%%%%%%%%%%%%%%%%%%%%%%%%%%%%%%%%%%%%%%%%%%%%%%%%%%%%%%%%%
%%                                                                 %%
%% For instructions on how to fill out this Tex template           %%
%% document please refer to Readme.html and the instructions for   %%
%% authors page on the biomed central website                      %%
%% http://www.biomedcentral.com/info/authors/                      %%
%%                                                                 %%
%% Please do not use \input{...} to include other tex files.       %%
%% Submit your LaTeX manuscript as one .tex document.              %%
%%                                                                 %%
%% All additional figures and files should be attached             %%
%% separately and not embedded in the \TeX\ document itself.       %%
%%                                                                 %%
%% BioMed Central currently use the MikTex distribution of         %%
%% TeX for Windows) of TeX and LaTeX.  This is available from      %%
%% http://www.miktex.org                                           %%
%%                                                                 %%
%%%%%%%%%%%%%%%%%%%%%%%%%%%%%%%%%%%%%%%%%%%%%%%%%%%%%%%%%%%%%%%%%%%%%

%%% additional documentclass options:
%  [doublespacing]
%  [linenumbers]   - put the line numbers on margins

%%% loading packages, author definitions

%\documentclass[twocolumn]{bmcart}% uncomment this for twocolumn layout and comment line below
\documentclass{bmcart}

%%% Load packages
%\usepackage{amsthm,amsmath}
%\RequirePackage{natbib}
%\RequirePackage[authoryear]{natbib}% uncomment this for author-year bibliography
%\RequirePackage{hyperref}
\usepackage[utf8]{inputenc} %unicode support
%\usepackage[applemac]{inputenc} %applemac support if unicode package fails
%\usepackage[latin1]{inputenc} %UNIX support if unicode package fails


%%%%%%%%%%%%%%%%%%%%%%%%%%%%%%%%%%%%%%%%%%%%%%%%%
%%                                             %%
%%  If you wish to display your graphics for   %%
%%  your own use using includegraphic or       %%
%%  includegraphics, then comment out the      %%
%%  following two lines of code.               %%
%%  NB: These line *must* be included when     %%
%%  submitting to BMC.                         %%
%%  All figure files must be submitted as      %%
%%  separate graphics through the BMC          %%
%%  submission process, not included in the    %%
%%  submitted article.                         %%
%%                                             %%
%%%%%%%%%%%%%%%%%%%%%%%%%%%%%%%%%%%%%%%%%%%%%%%%%


\def\includegraphic{}
\def\includegraphics{}



%%% Put your definitions there:
\startlocaldefs
\endlocaldefs


%%% Begin ...
\begin{document}

%%% Start of article front matter
\begin{frontmatter}

\begin{fmbox}
\dochead{Investigación}

%%%%%%%%%%%%%%%%%%%%%%%%%%%%%%%%%%%%%%%%%%%%%%
%%                                          %%
%% Enter the title of your article here     %%
%%                                          %%
%%%%%%%%%%%%%%%%%%%%%%%%%%%%%%%%%%%%%%%%%%%%%%

\title{Los sistema operativos para moviles (Android).}

%%%%%%%%%%%%%%%%%%%%%%%%%%%%%%%%%%%%%%%%%%%%%%
%%                                          %%
%% Enter the authors here                   %%
%%                                          %%
%% Specify information, if available,       %%
%% in the form:                             %%
%%   <key>={<id1>,<id2>}                    %%
%%   <key>=                                 %%
%% Comment or delete the keys which are     %%
%% not used. Repeat \author command as much %%
%% as required.                             %%
%%                                          %%
%%%%%%%%%%%%%%%%%%%%%%%%%%%%%%%%%%%%%%%%%%%%%%



%%%%%%%%%%%%%%%%%%%%%%%%%%%%%%%%%%%%%%%%%%%%%%
%%                                          %%
%% Enter the authors' addresses here        %%
%%                                          %%
%% Repeat \address commands as much as      %%
%% required.                                %%
%%                                          %%
%%%%%%%%%%%%%%%%%%%%%%%%%%%%%%%%%%%%%%%%%%%%%%



%%%%%%%%%%%%%%%%%%%%%%%%%%%%%%%%%%%%%%%%%%%%%%
%%                                          %%
%% Enter short notes here                   %%
%%                                          %%
%% Short notes will be after addresses      %%
%% on first page.                           %%
%%                                          %%
%%%%%%%%%%%%%%%%%%%%%%%%%%%%%%%%%%%%%%%%%%%%%%

\begin{artnotes}
%\note{Sample of title note}     % note to the article
\note[id=n1]{Equal contributor} % note, connected to author
\end{artnotes}

\end{fmbox}% comment this for two column layout

%%%%%%%%%%%%%%%%%%%%%%%%%%%%%%%%%%%%%%%%%%%%%%
%%                                          %%
%% The Abstract begins here                 %%
%%                                          %%
%% Please refer to the Instructions for     %%
%% authors on http://www.biomedcentral.com  %%
%% and include the section headings         %%
%% accordingly for your article type.       %%
%%                                          %%
%%%%%%%%%%%%%%%%%%%%%%%%%%%%%%%%%%%%%%%%%%%%%%

\begin{abstractbox}

\begin{abstract} % abstract

	El objetivo del articulo es dar a conocer el sistema operativo que hoy conocemos como "Android", que es comun mente utilizado por la mayoria de las personas, asi como dar a conocer la influencia del sistema operativo Android en el mundo de los dispositivos móviles inteligentes que en la actualidad han evolucionando de manera considerable.


\end{abstract}

%%%%%%%%%%%%%%%%%%%%%%%%%%%%%%%%%%%%%%%%%%%%%%
%%                                          %%
%% The keywords begin here                  %%
%%                                          %%
%% Put each keyword in separate \kwd{}.     %%
%%                                          %%
%%%%%%%%%%%%%%%%%%%%%%%%%%%%%%%%%%%%%%%%%%%%%%



% MSC classifications codes, if any
%\begin{keyword}[class=AMS]
%\kwd[Primary ]{}
%\kwd{}
%\kwd[; secondary ]{}
%\end{keyword}

\end{abstractbox}
%
%\end{fmbox}% uncomment this for twcolumn layout

\end{frontmatter}

%%%%%%%%%%%%%%%%%%%%%%%%%%%%%%%%%%%%%%%%%%%%%%
%%                                          %%
%% The Main Body begins here                %%
%%                                          %%
%% Please refer to the instructions for     %%
%% authors on:                              %%
%% http://www.biomedcentral.com/info/authors%%
%% and include the section headings         %%
%% accordingly for your article type.       %%
%%                                          %%
%% See the Results and Discussion section   %%
%% for details on how to create sub-sections%%
%%                                          %%
%% use \cite{...} to cite references        %%
%%  \cite{koon} and                         %%
%%  \cite{oreg,khar,zvai,xjon,schn,pond}    %%
%%  \nocite{smith,marg,hunn,advi,koha,mouse}%%
%%                                          %%
%%%%%%%%%%%%%%%%%%%%%%%%%%%%%%%%%%%%%%%%%%%%%%

%%%%%%%%%%%%%%%%%%%%%%%%% start of article main body
% <put your article body there>

%%%%%%%%%%%%%%%%
%% Background %%
%%
\newpage

\section*{Introducción}

Los sistemas operativos son muy comúnmente consocios hoy en día, porque cualquier persona ha tenido contacto con ellos, desde una computadora personal hasta algún dispositivo móvil e inclusive en algunos coches muy modernos, que cuentas con esta herramienta fundamental en la computación en informática.
La información que estos manejan es muy rápido, que no tiene comparación con cualquier otra cosa, esto es gracias a la evolución del hardware, que ha tenido en las últimas décadas, enfocándose siempre a la superación de la velocidad de procesamiento, y en la sencilles de para poder utilizarlos. \\

Tan solo 10 años los dispositivos moviles han ganado mucho terreno gracias a que son sencillo y faciles de utilizar con una interface grafica amigable con el usuario , esto demuestra que este tipo de tecnología evoluciona de manera considerable, muy rápidamente y constantemente tenemos nuevos avances en este campo de la tecnologia.\\

Hoy en día hablamos de sistemas que están enfocados a darle una experiencia al usuario de su uso cotidiano, y dejar de lado los comandos puros de un sistema en esencia puramente la comunicación con los componentes físicos de la maquina han dejado de ser cosa que solo expertos conozcan y dar al público esta accesibilidad que hoy tenemos.\\

El sistema operativo de Google "Android", no es la excepción, ya que se considera como unos de los sistemas operativos actuales en el mercado con muy alta demanda, una utilidad increible y un alto nivel de desarrollo para el futuro proximo.
%\cite{koon,oreg,khar,zvai,xjon,schn,pond,smith,marg,hunn,advi,koha,mouse}

\newpage

\section*{Definicion de un sistema operativo}

El sistema operativo es un grupo de programas de proceso con las rutinas de
control necesarias para mantener continuamente
operativos dichos programas.

\subsection*{Objetivo primario de un sistema operativo:}

El objetivo principal de un sistema operativo es optimizar todos los recursos del sistema para soportar los requerimientos que se le pidan.\cite{tanenbaum1998sistemas}\cite{luis2001sistemas}


\section*{“Smartphones” o teléfonos inteligentes}

Un “smartphone” (teléfono inteligente en español) es un dispositivo electrónico que
funciona como un teléfono móvil con características similares a las de una computadora
personal. Es como un equivalente entre un teléfono móvil clásico y una PDA
ya que permite hacer llamadas y enviar mensajes de texto como un móvil
convencional pero además incluye características cercanas a las de una computadora. 
Una característica importante de casi todos los teléfonos moviles es que
permiten la instalación de programas (aplicaciones) para incrementar el procesamiento de datos y la
conectividad. Estas aplicaciones pueden ser desarrolladas por el fabricante del
dispositivo, por el operador o por un tercero.
Los teléfonos inteligentes se distinguen por muchas características, entre las que
destacan las pantallas táctiles, un sistema operativo así como la conectividad a
Internet y el acceso al correo electrónico.\cite{del2009sistemas}


\section*{Localización en un sistema de computo}

El sistema operativo se encuentra por encima del lenguage maquina, y este es el principal traductor y conversor para el mismo.
Inferior a este se encuentra el hardware, y en el nivel mas bajo se encuentran los dispositivos fisicos hechos con circuitos integrados, cables, fuentes de potencia, etc.\cite{tanenbaum2003sistemas}


\section*{Principales Sistemas Operativos para dispositivos móviles}

Existen varios de ellos, si bien las más extendidas son Symbian, BlackBerry
OS, Windows Mobile, y recientemente iPhone OS y el sistema móvil de Google,
Android, además por supuesto de los dispositivos con sistema operativo Linux.\cite{alonsodispositivos}\cite{del2009sistemas}

\subsection*{Symbian:}

Este es el sistema operativo para móviles más extendido entre “smartphones”, y por
tanto el que más aplicaciones para su sistema tiene desarrolladas. Actualmente
Symbian copa más del 65 del mercado de sistemas operativos.\cite{del2009sistemas}

\subsubsection*{Ventajas:}

Su principal virtud es la capacidad que tiene el sistema para adaptar e integrar todo
tipo de aplicaciones. Admite la integración de aplicaciones y, como sistema operativo,
ofrece las rutinas, los protocolos de comunicación, el control de archivos y los
servicios para el correcto funcionamiento de estas aplicaciones.

\subsection*{Windows Mobile:}

Windows Mobile es un sistema operativo escrito desde 0 y que hace uso de
algunas convenciones de la interfaz de usuario del Windows de siempre.\cite{del2009sistemas}

\subsubsection*{Ventajas:}

Una de las ventajas de Windows Mobile sobre sus competidores es que los
programadores pueden desarrollar aplicaciones para móviles utilizando los mismos
lenguajes y entornos que emplean con Windows para PC. En comparación, las
aplicaciones para Symbian necesitan más esfuerzo de desarrollo, aunque también
están optimizadas para cada modelo de teléfono.


\subsection*{Android:}

Android es un sistema operativo móvil basado en Linux y Java que ha sido
liberado bajo la licencia Apache versión 2.
El sistema busca, nuevamente, un modelo estandarizado de programación que
simplifique las labores de creación de aplicaciones móviles y normalice las
herramientas en el campo de la telefonía móvil.\cite{del2009sistemas}\cite{polanco2011android}

\subsubsection*{Ventajas:}

Su principal ventaja es que los programadores sólo tengan que desarrollar sus creaciones
una única vez y así ésta sea compatible con diferentes terminales gracias a la alta compatibilidad del leguage de programacion Java. 


\subsection*{ iPhone OS:}

Es una versión reducida de Mac OS X optimizada para los procesadores
ARM. Aunque oficialmente no se puede instalar ninguna aplicación que no esté
firmada por Apple ya existen formas de hacerlo, la vía oficial forma parte del iPhone
Developer Program (de pago) y hay que descargar el SKD que es gratuito.\cite{del2009sistemas}

\subsubsection*{Ventajas:}

iPhone dispone de un interfaz de usuario realmente interesante, lo unico que no satisface es la
cantidad de restricciones que tiene. para que iOS
triunfe mucho más es mejor liberar y dar libertad a su sistema.


\subsection*{ Blackberry OS:}

BlackBerry es un sistema operativo multitarea que está arrasando en la escena
empresarial, en especial por sus servicios para correo y teclado QWERTY.
Actualmente BlackBerry OS cuenta con un 11\% del mercado.\cite{del2009sistemas}

\subsubsection*{Ventajas:}
Este sistema operativo incorpora múltiples aplicaciones y programas
que convierten a los dispositivos en completos organizadores de bolsillo con
funciones de calendario, libreta de direcciones, bloc de notas, lista de tareas, entre
otras.



\newpage  

\section*{Android}

\subsection*{Historia}
En sus inicios, únicamente trascendió que la actividad de la empresa se
centraba en “el desarrollo de software para teléfonos móviles“.\newline
Android Inc. pasó casi dos años trabajando “en la sombra”, hasta que Google comenzó
a “reclutar” a fuerza de talonario a algunas “startup” (término que se refiere a nuevas
compañías con un futuro prometedor) del sector móvil, con la clara intención de replicar su
éxito de la Web en el futuro de las telecomunicaciones inalámbricas.

En Mayo del mismo año Google se hacía con Dodgeball, la empresa que desarrolló un
sistema de red social y posicionamiento móvil que, una vez integrada en la estructura
empresarial de los chicos de Mountain View, cesó su actividad en 2009 para dar paso a Google
Latitude. Llegó el mes de agosto y le tocó el turno a Android Inc., la fecha clave para llegar a
entender mejor el éxito de Android es el 5 de noviembre de 2007. Ese día se fundaba la OHA
(Open Handset Alliance), una alianza comercial de 35 componentes iniciales liderada por
Google, que contaba con fabricantes de terminales móviles, operadores de
telecomunicaciones, fabricantes de chips y desarrolladores de software. El mismo día se dio a
conocer por vez primera lo que hoy conocemos como Android, una plataforma de código
abierto para móviles que se presentaba con la garantía de estar basada en el sistema operativo
Linux. 

En Mayo del mismo año Google se hacía con Dodgeball, la empresa que desarrolló un
sistema de red social y posicionamiento móvil que, una vez integrada en la estructura
empresarial de los chicos de Mountain View, cesó su actividad en 2009 para dar paso a Google
Latitude. Llegó el mes de agosto y le tocó el turno a Android Inc., la fecha clave para llegar a
entender mejor el éxito de Android es el 5 de noviembre de 2007. Ese día se fundaba la OHA
(Open Handset Alliance), una alianza comercial de 35 componentes iniciales liderada por
Google, que contaba con fabricantes de terminales móviles, operadores de
telecomunicaciones, fabricantes de chips y desarrolladores de software. El mismo día se dio a
conocer por vez primera lo que hoy conocemos como Android, una plataforma de código
abierto para móviles que se presentaba con la garantía de estar basada en el sistema operativo
Linux. \cite{herraiz2012android}\cite{baez1997introduccion}\cite{benbourahala2013android}\newline


\subsection*{¿Que es Android?}

Android es un sistema operativo basado en GNU/Linux de código abierto bajo licencia Apache, el cual permite la creación principalmente de aplicaciones para dispositivos móviles teléfonos inteligentes, tablets, reproductores MP3, notebook, y otros desarrollado por Google  y actualmente liderado por el grupo Open Handset Alliance, en el cual se agrupan varias compañías del sector, entre las cuales se encuentran: Google, Samsung, HTC, Dell, Intel, Qualcomm, Motorola, LG, Telefónica, T-Mobile, Nvidia.\cite{vanegas2014android}\cite{polanco2011android}\cite{baez1997introduccion}\cite{benbourahala2013android}


\subsection*{Arquitectura:}

La arquitectura de Android, se conforma por cuatro capas o niveles que le permiten al programador la creación de aplicaciones. Su distribución ayuda a acceder a las diferentes capas por medio de librerías y cada capa utiliza los elementos de la capa inferior para realizar sus funciones, por eso, su arquitectura es tipo pila. La arquitectura del sistema operativo Android se puede aprecia en la siguiente gráfica:\cite{vanegas2014android}\newline

Kernel de Linux.
El núcleo actúa como una capa de abstracción entre el hardware y el resto de las capas de la arquitectura. 
\newline


Librerías.
La siguiente capa que se sitúa justo sobre el kernel la componen las bibliotecas nativas de Android, también llamadas librerías. Están escritas en C o C++ y compiladas para la arquitectura hardware específica del teléfono.
\newline

Entorno de ejecución.
Aquí encontramos las librerías con la funcionalidades habituales de Java así como otras específicas de Android.
\newline

Framework de aplicaciones.
La siguiente capa está formada por todas las clases y servicios que utilizan directamente las aplicaciones para realizar sus funciones. 
\newline

\subsection*{Versiones:}
Cupcake: Android Version 1.5\newline
Donut: Android Version 1.6 \newline
Eclair: Android Version 2.0/2.1 \newline
Froyo: Android Version 2.2\newline
Ginger Bread: Android Version 2.3 \newline
Honey Comb: Android Version 3.0/3.4\newline
Ice Cream Sandwich: Android Version 4.0 \newline
Jelly Bean: Android Version 4.1 \newline
Jelly Bean (Gummy Bear): Android Version 4.2 \newline
Jelly Bean: Android Version 4.3 \newline
KitKat (Dugger): Android Version 4.4 \newline
Lollipop: Android Version 5.0\newline
Marshmallow: Android Version 6.0


\newpage

%%%%%%%%%%%%%%%%%%%%%%%%%%%%%%%%%%%%%%%%%%%%%%
%%                                          %%
%% Backmatter begins here                   %%
%%                                          %%
%%%%%%%%%%%%%%%%%%%%%%%%%%%%%%%%%%%%%%%%%%%%%%

\begin{backmatter}


%%%%%%%%%%%%%%%%%%%%%%%%%%%%%%%%%%%%%%%%%%%%%%%%%%%%%%%%%%%%%
%%                  The Bibliography                       %%
%%                                                         %%
%%  Bmc_mathpys.bst  will be used to                       %%
%%  create a .BBL file for submission.                     %%
%%  After submission of the .TEX file,                     %%
%%  you will be prompted to submit your .BBL file.         %%
%%                                                         %%
%%                                                         %%
%%  Note that the displayed Bibliography will not          %%
%%  necessarily be rendered by Latex exactly as specified  %%
%%  in the online Instructions for Authors.                %%
%%                                                         %%
%%%%%%%%%%%%%%%%%%%%%%%%%%%%%%%%%%%%%%%%%%%%%%%%%%%%%%%%%%%%%

% if your bibliography is in bibtex format, use those commands:
\bibliographystyle{bmc-mathphys} % Style BST file (bmc-mathphys, vancouver, spbasic).
\bibliography{bmc_article}      % Bibliography file (usually '*.bib' )
% for author-year bibliography (bmc-mathphys or spbasic)
% a) write to bib file (bmc-mathphys only)
% @settings{label, options="nameyear"}
% b) uncomment next line
%\nocite{label}

% or include bibliography directly:
% \begin{thebibliography}
% \bibitem{b1}
% \end{thebibliography}

%%%%%%%%%%%%%%%%%%%%%%%%%%%%%%%%%%%
%%                               %%
%% Figures                       %%
%%                               %%
%% NB: this is for captions and  %%
%% Titles. All graphics must be  %%
%% submitted separately and NOT  %%
%% included in the Tex document  %%
%%                               %%
%%%%%%%%%%%%%%%%%%%%%%%%%%%%%%%%%%%

%%
%% Do not use \listoffigures as most will included as separate files


%%%%%%%%%%%%%%%%%%%%%%%%%%%%%%%%%%%
%%                               %%
%% Tables                        %%
%%                               %%
%%%%%%%%%%%%%%%%%%%%%%%%%%%%%%%%%%%

%% Use of \listoftables is discouraged.
%%


%%%%%%%%%%%%%%%%%%%%%%%%%%%%%%%%%%%
%%                               %%
%% Additional Files              %%
%%                               %%
%%%%%%%%%%%%%%%%%%%%%%%%%%%%%%%%%%%


\end{backmatter}
\end{document}
